% Do not change the options here
\documentclass[bsc,frontabs,singlespacing,parskip,deptreport,normalheadings]{infthesis}

\usepackage{amsmath}

\begin{document}
\begin{preliminary}

\title{This is the Project Title}

\author{Neil Weidinger}

% to choose your course
% please un-comment just one of the following
% \course{Artificial Intelligence}
%\course{Artificial Intelligence and Computer Science}
%\course{Artificial Intelligence and Mathematics}
%\course{Artificial Intelligence and Software Engineering}
%\course{Artificial Intelligence with Management}
%\course{Cognitive Science}
\course{Computer Science}
%\course{Computer Science and Management Science}
%\course{Computer Science and Mathematics}
%\course{Computer Science and Physics}
%\course{Computer Science with Management}
%\course{Software Engineering}
%\course{Software Engineering with Management}

\project{4th Year Project Report}

\date{\today}

\abstract{
This skeleton demonstrates how to use the \texttt{infthesis} style for
undergraduate dissertations in the School of Informatics. It also emphasises the
page limit, and that you must not deviate from the required style.
The file \texttt{skeleton.tex} generates this document and can be used as a
starting point for your thesis. The abstract should summarise your report and
fit in the space on the first page.
}

\maketitle

\section*{Acknowledgements}
Acknowledgements go here.

\tableofcontents

\end{preliminary}

%%%%%%%%%%%%%%%%%%%%%%%%%%%%%%%%%%%%%%%%%%%%%%%%%%%%%%%%%%%%%%%%%%%%%%%%%%%%%%%%

\chapter{Introduction}

\section{Motivation}

\section{Contributions}

\section{Report outline}

%%%%%%%%%%%%%%%%%%%%%%%%%%%%%%%%%%%%%%%%%%%%%%%%%%%%%%%%%%%%%%%%%%%%%%%%%%%%%%%%

\chapter{Background}

Describe and outline background chapter.

\section{Modern computer architecture}

Ever since the advent of general-purpose microprocessor based computer systems,
transistor density has been doubling roughly every two years. Famously known as
Moore's law, for the first 30 years of the existence of the microprocessor the
consequences of this graced the computing world with effortless biennial
performance increases.  Bestowed with this exponential growth of transistor
density, chip designers could drastically increase core frequencies with each
generation, and with ever increasing transistor budgets afford to design more
complex architectural features like instruction pipelining, superscalar
execution, and branch prediction. Without touching a line of code, software
developers could expect programs to automatically double in performance every
two years \cite{hennessy_new_2019}.

Accompanying Moore's law was another, related, effect: Dennard scaling. While
Moore's law provided increased transistor counts, Dennard scaling allowed for
this transistor doubling while ensuring power density was constant. The scaling
law states roughly that as transistors get smaller, power density stays
constant, meaning that power consumption with double the transistors stays the
same. Additionally, as transistor sizes scale downward, the reduced physical
distances enable reduced circuit delays, meaning an increase in clock frequency,
boosting chip performance. When combined, with every technology generation
transistor densities double, clock speeds increase by roughly 40\%, and power
consumption remains the same \cite{borkar_future_2011}. This remarkable scaling
is what historically allowed for incredible performance gains year over year,
all while keeping a reasonable energy envelope.

But starting around 2005, Dennard scaling has broken down: processors have
reached the physical limits of power consumption in order to avoid thermal
runaway effects that would require prohibitive cooling solutions. This is known
as the power wall, and chip designers could no longer regularly rely on
increasing clock frequencies to deliver performance gains
\cite{parkhurst_single_2006}. The multicore era was born.

\subsection{Rise of the multicore era}

Thankfully, Moore's law is still going strong, but instead of being dedicated to
more complex architectural features in order to extract sequential performance
in single core processors, the extra transistors are largely used to build more
cores. With diminishing returns on effort spent increasing single core
performance, chip designers look to add multiple cores to be able to execute
more instructions in parallel.

CPU performance can be described using the following equation
\cite{patterson_computer_2021}, where performance is measured in terms of
absolute execution time: \[ \text{CPU execution time for a program} =
\frac{\text{Program instruction count} \cdot \text{CPI}}{\text{Clock rate
(frequency)}} \]

where CPI is average clock cycles per instruction. No longer able to increase
the clock frequency due to the power wall and increasing difficulty reducing
CPI, efforts of hardware architects focused on simply reducing program
instruction count per processor, by distributing instructions across multiple
CPU cores to be executed in parallel.

Multicore processors are microprocessors containing multiple processors in a
single integrated circuit, where each of these processors is known as a core.
Almost all commodity multicore processors today are \textit{symmetric
multiprocessing (SMP)} systems, meaning all cores in a processor share the same
physical address space. A single physical address space allows cores to operate
on shared data. Armed with multiple cores, different programs or different parts
of the same program can be run at the same time in parallel, reducing the time
required to perform the same amount of work on a single processor, boosting
performance.

Other variations of the multiprocessor philosophy (essentially running workloads
on multiple processors in parallel) can be found in other approaches to parallel
computing, for example network connected cluster based designs often found in
the High Performance Computing (HPC) domain. These approaches will not be
discussed in this report, although the ideas described could possibly be
transferable.

As of 2021, it is difficult to find a processor that is not a multicore
processor. The performance gains provided by having multiple cores have shown to
be so profound that even the lowest end chips feature multiple cores. The
recently released Raspberry Pi Zero 2, a \pounds13.50 board in the Raspberry Pi
family of low cost single-board computers, features a 64-bit quad-core Arm Cortex-A53
CPU.

Initially multicore processors may seem like a silver bullet to the question of
what to do when faced with the power wall: for more performance simply scale the
number of cores! In reality, as is typical, the situation is more nuanced. Many
workloads cannot be trivially diced up and processed on multiple cores, and even
if so, support for splitting up work and then computing this work in parallel
must be explicitly supported and designed for.

\subsection{Parallel computing and its difficulties}

Even if the hardware a given piece of software is running on features multiple
cores that can be run in parallel, the software must be carefully designed such
that it actually takes advantage of the multiple processing units. 

\subsubsection{Amdahl's Law}

\section{Classical work stealing}

\section{Survey of related work}

%%%%%%%%%%%%%%%%%%%%%%%%%%%%%%%%%%%%%%%%%%%%%%%%%%%%%%%%%%%%%%%%%%%%%%%%%%%%%%%%

\chapter{Conclusions}

\section{Final Reminder}

The body of your dissertation, before the references and any appendices,
\emph{must} finish by page~40. The introduction, after preliminary material,
should have started on page~1.

You may not change the dissertation format (e.g., reduce the font size, change
the margins, or reduce the line spacing from the default 1.5 spacing). Be
careful if you copy-paste packages into your document preamble from elsewhere.
Some \LaTeX{} packages such as \texttt{geometry}, \texttt{fullpage}, or
\texttt{savetrees} change the margins of your document. Do not include them!

Over length or incorrectly-formatted dissertations will not be accepted and you
would have to modify your dissertation and resubmit. You cannot assume we will
check your submission before the final deadline and if it requires resubmission
after the deadline to conform to the page and style requirements you will be
subject to the usual late penalties based on your final submission time.

\bibliographystyle{plain}
\bibliography{references}

%% You can include appendices like this:
% \appendix
%
% \chapter{First appendix}
%
% \section{First section}
%
% Markers do not have to consider appendices. Make sure that your contributions
% are made clear in the main body of the dissertation (within the page limit).

\end{document}
